% !TeX TS-program = xelatex
% !TeX builder = latexmk
% !BIB program = biblatex

% XeLaTeX stuff:
% Normalize any residual Unicode combining accents and write out error messages
\XeTeXinputnormalization=1
\tracinglostchars=1
\tracingonline=1
\XeTeXgenerateactualtext=1

% Don't modify the `\DocumentMetadata` command unless you know what it does!
% If this command throws an "undefined" error, your LaTeX system is out of date.
\DocumentMetadata{
  pdfstandard = a-2b,
  pdfversion  = 1.7,
  lang		    = en-US
}

\documentclass[twoside]{xarticle}

\setcounter{tocdepth}{3}
\setcounter{secnumdepth}{0}

\title{Concepts, Elements and Components of Mantra Japa Vidhi}
\author{\relax}
\date{\relax}

\hypersetup{%
  pdftitle={\relax},
  pdfauthor={\relax},
  pdfauthortitle={\relax},
	pdfsubject={XeLaTeX},
	pdfkeywords={latex,xelatex,typesetting}
}

\begin{document}
\hyphenation{%
  deva-nāgarī
  deva-nagari
  śaṅ-karā-cārya
  san-s-krit
}

\thispagestyle{empty}
\maketitle
\tableofcontents

\clearpage

\pagestyle{fancy}
\thispagestyle{empty}

\begin{abstract}\noindent
  This article is a glossary of Tantric terms and a simple answer to questions
  related to \sa{mantra japa}. For those of us adopting the path of
  \sa{mantra japa} and learning its nuances, a basic elementary understanding
  is required and that is the main objective of this article.
\end{abstract}

\section{Viniyogaḥ}
Deployment or the purpose of performing the \sa{mantra japa} of the deity.
Generally, the words associated with a \sa{viniyogaḥ} are
‘\sa{abhīṣṭa siddhyarthe}’, ‘\sa{caturvarga siddhyarthe}’,
‘\sa{devatā prītyarthe}’, ‘\sa{darśanabhāgya siddhyarthe}’,
‘\sa{gururājñāya}’, \etc

\begin{enumerate}
  \item \sa{abhīṣṭa siddhyarthe} (\dn{अभीष्ट सिद्ध्यर्थे}): directly translates to wish fulfillment of all cherished desires.
  \item \sa{caturvarga siddhyarthe} (\dn{चतुर्वर्ग सिद्ध्यर्थे}): refers to the fulfillment of the four cherished objectives also called as puruṣārthas: \sa{dharma} (righteousness), \sa{artha} (wealth), \sa{kāma} (desires) and \sa{mokṣa} (liberation).
  \item \sa{devatā prītyarthe} (\dn{देवता प्रीत्यर्थे}): translates to pleasing the deity whose \sa{mantra japa} is being performed or contemplated upon.
  \item \sa{darśanabhāgya siddhyarthe} (\dn{दर्शनभाग्य सिद्ध्यर्थे}): refers to obtaining the vision of the deity that is being contemplated upon for the \sa{mantra japa}.
  \item \sa{gurūrājñāya} (\dn{गुरूराज्ञाय}): As ordained by the \sa{guru}.
\end{enumerate}

The components of the \sa{viniyoga} such as the sage (\sa{ṛṣi}),
meter (\sa{chandas}), seed (\sa{bīja}), power (\sa{śakti}) and pin (\sa{kīlakam})
are visualized and performed as part of the \sa{ṛṣyādi nyāsa}.

\section{Ṛṣyādi nyāsa}

The procedure of deploying the \sa{viniyoga} in our body is called
the \sa{ṛṣyādi nyāsa}. It is the embodiment of the \sa{viniyoga} within
ourselves before the commencement of the \sa{mantra japa}.

\subsection{Ṛṣiḥ}

\textbf{ṛṣaye namaḥ śirasi} (\dn{ऋषये नमः शिरसि})\\
The sage or seer who has discovered the mantra and its utility after gaining
\sa{siddhi} (fruition) and has generously disclosed it for the benefit of
mankind, is the \sa{ṛṣi} of the mantra. He/She is to be placed on the top of
our head with the right palm placed downwards or touching the top of our head
with the middle and ring finger joined together.

The top of the head represents the \sa{sahasrāra}, the orifice of the causal
body which interfaces with the Divine super-consciousness present everywhere.
The grace of the Divine super-consciousness is set to descend from
the \sa{sahasrāra} into the body and activate the rest of the six \sa{cakra}-s
present in the body. The realized \sa{ṛṣi} who has revealed the mantra,
is elevated to the state of the Divine super-consciousness and we urge him/her
to grant their grace upon us and help us realize the benefits of the mantra
just as they did for themselves! The onus and responsibility to follow
the path of the \sa{ṛṣi} and that of the \sa{guru pāduka} (Guru’s words and
tradition) is upon us now!

\subsection{Chandaḥ}

\textbf{chandase namaḥ mukhe} (\dn{छन्दसे नमः मुखे})\\
The Vedic meter or prosody that applies to the mantra in consideration. The word
‘\sa{chandas}’ typically conceals the metrical composition of the mantra itself
or its \sa{uddhāra śloka} (The verse that reveals the composition and make-up
of the mantra). The pronunciation or the rhyme associated with the mantra would
resonate with the \sa{chandas} that is mentioned in the \sa{viniyoga}.
The secrets and revelation of the mantra are encoded within the \sa{chandas}.
The mouth is covered with four fingers and the nose is touched to propitiate
the \sa{chandas}. It indicates that the secrets of the \sa{chandas} cannot be
expressed in words and also to indicate that the \sa{mantra japa} itself should
align with the breath and recited mentally and not verbally.

\subsection{Devatā}

\textbf{devatāyai namaḥ hṛdi} (\dn{देवतायै नमः हृदि})\\
The Devata or deity associated with the mantra is propitiated by placing
the index, middle and ring fingers joined together on the heart. It signifies
that we hold the deity in our heart and wish to forge a permanent bond with
deep love, complete trust, surrender and affection.

\subsection{Bīja}

\textbf{bījāya namaḥ guhye} (\dn{बीजाय नमः गुह्ये})\\
\sa{Bīja} is the seed of the mantra and all its intended benefits and secrets,
are encapsulated within. The \sa{bīja} of the mantra is to placed on
the genitals using the \sa{tattva mudra} of the thumb and ring finger joining
together.

\subsection{Śakti}

\textbf{śaktaye namaḥ pādayoḥ} (\dn{शक्तये नमः पादयोः})\\
\sa{Śakti} is the power component associated with the mantra and the associated
seed syllable(s) are responsible for manifesting the results of the mantra.
The ‘\sa{śakti}’ is to be placed on the feet and is analogous with gaining
the power of the mantra and incorporating the same in all walks of life.

\subsection{Kīlaka}

\textbf{kīlakāya namaḥ nābhau} (\dn{कीलकाय नमः नाभौ})\\
\sa{Kīlaka} is the wedge or pin to which the mantra is fixed to. The effects of
the mantra are pinned to the \sa{kīlaka} and are therefore completely
influenced by it. It can also be seen as the flavor that applies to the results.
The \sa{kīlaka} is to be placed on the navel by means of the \sa{tattva mudra}.
The navel represents the spiritual umbilical connection with the deity and
reinforcing the same has a direct correspondence to the benefits obtained by
means of the \sa{mantra japa}. The navel \sa{maṇipūraka cakra} boosts the fire
and resolve within us to linger on through the karmic maze and realize all
the desired and promised benefits of the mantra.

\subsection{Viniyoga}

\textbf{jape viniyogāya namaḥ sarvāṅge} (\dn{जपे विनियोगाय नमः सर्वाङ्गे})\\
The entire \sa{viniyoga} is applied from the top to the bottom of the body by
sliding the hands from the top of the head to the feet on the sides our body or
by making a circle from top to bottom with the palms stretched out. This is to
indicate our total involvement in carrying out the \sa{mantra japa} and
realizing its full potential.

\clearpage

\appendix

\section{Appendix I}
\thispagestyle{empty}

\end{document}
