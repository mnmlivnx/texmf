% \def\filename{brahmic.dtx}
% \def\fileversion{0.01 beta}
% \def\filedate{2015/06/30}
%
% \iffalse meta-comment
%
% David M. Jones
% email: dmj@ams.org
%
% Copyright 2015 David M. Jones
%
% This work may be distributed and/or modified under the
% conditions of the LaTeX Project Public License, either version 1.3c
% of this license or (at your option) any later version.
% The latest version of this license is in
%   http://www.latex-project.org/lppl.txt
% and version 1.3c or later is part of all distributions of LaTeX
% version 2005/12/01 or later.
% 
% This work has the LPPL maintenance status `maintained'.
% 
% The Current Maintainer of this work is David M. Jones.
%
% \fi
%
% \iffalse
%<*driver>
\NeedsTeXFormat{LaTeX2e}
\documentclass{ltxdoc}

\setcounter{tocdepth}{2}
\setcounter{secnumdepth}{2}

\usepackage{hyperref}

\usepackage{metalogo}

\usepackage{fontspec}

% Brill 2.06: http://www.brill.com/about/brill-fonts

\setmainfont[Numbers=Lining]{Brill}

% No endorsement is meant for any of these fonts (excerpt for Murty
% Hindi).  I simply looked for the first font that had all or most of
% the characters I needed.  I'm open to suggestions for replacement
% fonts, especially free alternatives to Arial Unicode MS.

% Murty Hindi 1.01: http://www.murtylibrary.com/mcli-fonts.php

\newfontfamily\dev[Script=Devanagari,Mapping=deva]{Murty Hindi}
\newfontfamily\san[Script=Devanagari,Mapping=deva-san]{Murty Hindi}

\newfontfamily\beng[Mapping=beng]{SolaimanLipi}
\newfontfamily\bengalt[Mapping=beng]{AponaLohit}
\newfontfamily\gujr[Mapping=gujr]{Rekha}
\newfontfamily\guru[Mapping=guru]{AnmolUni}
\newfontfamily\knda[Mapping=knda]{Arial Unicode MS}
\newfontfamily\mlym[Mapping=mlym]{Arial Unicode MS}
\newfontfamily\orya[Mapping=orya]{Arial Unicode MS}
\newfontfamily\sinh[Mapping=sinh]{Akshar Unicode}
\newfontfamily\taml[Mapping=taml]{Kadambri}
\newfontfamily\telu[Mapping=telu]{Arial Unicode MS}

\usepackage{sanskrit}

\renewenvironment{quote}
               {\list{}{\rightmargin0pt }%
                \item\relax}
               {\endlist}

\usepackage{longtable}
\usepackage{array}

\newcommand{\A}[1]{%
    #1&#1&#1&#1&#1&#1&#1&#1&#1&#1&#1&#1\\
}

\newcommand{\NT}[1]{% Not Tamil
    #1&#1&#1&#1&#1&#1&#1&&#1&#1&#1&#1\\
}

\newcommand{\SINH}[1]{%
    #1&&&&&&&&&&&#1\\
}

\def\B#1{\multicolumn{#1}{l}{}}

\def\L{\normalfont}
\renewcommand{\S}{\textbf}
\begin{document}
\DeleteShortVerb{\|}
\title{Brahmic TECkit mappings}
\author{David M. Jones}
\date{Version \fileversion, \filedate}
\DocInput{brahmic.dtx}
\end{document}
%</driver>
% \fi
%
% \maketitle
%
% \begin{abstract}
% This file defines a set of TECkit mappings based on ISO 15919
% \emph{Transliteration of Devanagari and related Indic scripts into
%   Latin characters} for reverse transliteration of Latin-based
% characters into various Indic scripts in the Brahmic family.
% \end{abstract}
%
% \tableofcontents
%
% \section{Introduction}
%
% This grew out of my experiments with typsetting Sanskrit text using
% \XeLaTeX, which in turn stemmed from my attempts to satisfy my
% curiosity about how the Devanagari script works.  With these
% mappings and suitable macros, you can write\footnote{Quoted from
% section~20 of \emph{A Sanskrit Grammar for Students}\cite{sgs}}
% \begin{verbatim}
% A simple \textbf{liquid vowel} followed by any \emph{other} vowel or
% by a diphthong is changed into its semivowel;—e.g.\ \N{dadhi atra} =
% \NS{dadhy atra}; \N{kartṛ uta} = \NS{kartr uta}; \N{madhu iva} = \NS{madv
%   iva}; \N{nadī artham} = \N{nadyartham}.\end{verbatim}
% and get
% \begin{quote}
% A simple \textbf{liquid vowel} followed by any \emph{other} vowel or
% by a diphthong is changed into its semivowel;—e.g.\ \N{dadhi atra} =
% \NS{dadhy atra}; \N{kartṛ uta} = \NS{kartr uta}; \N{madhu iva} = \NS{madv
%   iva}; \N{nadī artham} = \N{nadyartham}.
% \end{quote}
% Since the same input is used to generate both the Roman and the
% Devanagari text, this eliminates the possibility that a typo in one
% or the other will cause a mismatch—assuming the mapping is implemented
% correctly, of course!
%
% \subsection{Acknowledgements}
%
% The first draft map was based on version~1.2 of
% \texttt{velthuis-sanskrit.map}.  I've mauled the code almost beyond
% recognition, but it's influence is still apparent throughout
% Phase~2.
%
% \subsection{Caveats}
%
% I'm fairly confident that the Devanagari transliteration
% works,\footnote{At least, it did before I started adding re-arranging
% things to add other scripts.} but I'm much less sanguine about
% the other scripts, of which I lack even the modest knowledge I have
% about Devanagari.  I'd be grateful for any feedback from more
% knowledgeable users.
%
% \subsection{Supported scripts}
%
% \begin{tabular}{@{}ll}
% deva & Devanagari\\
% beng & Bengali\\
% gujr & Gujarati\\
% guru & Gurmukhi\\
% knda & Kannada\\
% mlym & Malayalam\\
% orya & Oriya\\
% sinh & Sinhala\\
% taml & Tamil\\
% telu & Telugu
% \end{tabular}
%
% \section{Transcription table}
%
% This table is adapted from~\cite{iso15919}.  The main (intentional)
% departures are in the rows for \S{e}, \S{ē}, \S{ê}, \S{o}, \S{ō},
% \S{ô}, and \S{ṃ}.  Strictly speaking, ISO~15919 only allows them in
% selected scripts, but for convenience, I've allowed them to be used
% in all scripts as synonyms for something similar.
%
% For purposes of illustration, I tried to find fonts that contain all
% of the characters covered, but there are still some missing glyphs
% in Devanagari, Kannada, Malayalam, Oriya, Telugu.  I'd be grateful
% for suggestions of fonts containing those characters.

% \begin{longtable}{@{}
%     >{\bfseries}l
%     >{\dev}l
%     >{\san}l
%     >{\guru}l
%     >{\gujr}l
%     >{\beng}l
%     >{\orya}l
%     >{\taml}l
%     >{\mlym}l
%     >{\knda}l
%     >{\telu}l
%     >{\sinh}l
%     @{}}
% \L Latin
% & \L Dev.
% & \L San.
% & \L Gur.
% & \L Guj.
% & \L Ben.
% & \L Ori.
% & \L Tam.
% & \L Mal.
% & \L Kan.
% & \L Tel.
% & \L Sin.
% % \endhead
% 
% \A{a}
% \A{ā}
% \SINH{æ}
% \SINH{ǣ}
% \A{i}
% \A{ī}
% \A{u}
% \A{ū}
% ŭ & \B{7} & ŭ\\
% r̥  & r̥  & r̥  & & r̥  & r̥  & r̥  & & r̥  & r̥  & r̥  & r̥ \\
% 
% r̥̄ & r̥̄ & r̥̄ & & r̥̄ & r̥̄ & r̥̄ & & r̥̄ & r̥̄ & r̥̄ & r̥̄ \\
% 
% l̥  & l̥  & l̥  & & l̥  & l̥  & l̥  & & l̥  & l̥  & l̥  & l̥  \\
% 
% l̥̄ & l̥̄ & l̥̄ & & l̥̄ & l̥̄ & l̥̄ & & l̥̄ & l̥̄ & l̥̄ & l̥̄\\[\smallskipamount]
% 
% \A{e}
% \A{ē}
% \A{ê}
% \A{ai}
% \A{o}
% \A{ō}
% \A{ô}
% \A{au}[\smallskipamount]
% \NT{ṁ}
% \NT{ṃ}[\jot]
% m̐ & m̐ & m̐  & m̐  & m̐  & m̐  & m̐  & \B{3} & m̐ \\
% n̆&&&&&&&&&&n̆\\
% \NT{ḥ}
% ẖ &  ẖ & ẖ &&&&&&&ẖ\\
% ḫ &  ḫ & ḫ &&&&&&&ḫ\\
% ḵ & &&&&&& ḵ\\[2\smallskipamount]
% 
% \A{ka}
% \NT{kha}
% \NT{ga}
% \NT{gha}
% \A{ṅa}
% \SINH{n̆ga}[\smallskipamount]
% 
% \A{ca}
% ĉa&&&&&&&&&&ĉa\\
% \NT{cha}
% \A{ja}
% \NT{jha}
% \A{ña}
% \SINH{n̆ja}[\smallskipamount]
% 
% \A{ṭa}
% \NT{ṭha}
% \NT{ḍa}
% ṛa & ṛa & & ṛa && ṛa & ṛa\\
% \NT{ḍha}
% ṛha & ṛha & & && ṛha & ṛha\\
% \A{ṇa}
% \SINH{n̆ḍa}[\smallskipamount]
% 
% \A{ta}
% \NT{tha}
% \NT{da}
% \NT{dha}
% \A{na}
% \SINH{n̆da}[\smallskipamount]
% 
% \A{pa}
% \NT{pha}
% \NT{ba}
% \NT{bha}
% \A{ma}
% \SINH{m̆ba}[\smallskipamount]
% 
% ṟa & ṟa & \B{5} & ṟa & ṟa & ṟa & ṟa & ṟa \\
% ṯa & \B{7} & ṯa\\
% ṉa & ṉa & \B{5} & ṉa & ṉa & \B{2} & ṉa \\
% ḻa & ḻa & \B{5} & ḻa & ḻa & ḻa & ḻa & ḻa\\[\smallskipamount]
% 
% \A{ya}
% ẏa & ẏa & \B{3} & ẏa & ẏa \\
% \A{ra}
% r̆ & r̆ & r̆ \\
% \A{la}
% ḷa & ḷa & ḷa & ḷa & ḷa  & & ḷa & ḷa & ḷa & ḷa & ḷa & ḷa \\
% \A{va}[\smallskipamount]
% 
% \A{śa}
% ṣa & ṣa & ṣa  & & ṣa & ṣa & ṣa & ṣa & ṣa & ṣa & ṣa & ṣa \\
% \A{sa}
% \A{ha}[\smallskipamount]
% ' & ' & ' & & ' & \bengalt' & ' & & ' & ' & ' \\[\smallskipamount]
% 
% qa&qa&&qa&qa&qa\\
% ḵẖa&ḵẖa&&ḵẖa&ḵẖa&ḵẖa\\
% ġa&ġa&&ġa&ġa&ġa\\
% za&za&&za&za&za&\B{3}&za & za\\
% fa&fa&&fa&fa&fa&\B{3}&fa&&fa\\
% wa&\B{4}&wa&wa\\
% 
% \end{longtable}
%
% \subsection{Special conventions}
%
% \begin{itemize}\parskip0pt  \itemsep0pt
%
% \item A colon (:) can be used to break a ligature or diphthong:
%   \N{b:ha} vs.\ \N{bha}, \N{pra:u} vs.\ \N{prau}.
%
% \item Use U+007C VERTICAL LINE for dandas: \D{|}, \D{||}.
%
% \item Use U+00B0 DEGREE SIGN for U+0970 DEVANAGARI ABBREVIATION SIGN
%
% \end{itemize}
%
% \subsection{Special extensions for Sanskrit}
%
% \begin{itemize}\parskip0pt \itemsep0pt
%
% \item \S{ṛ} can be used instead of \S{r̥} for vocalic~r.
%
% \item \S{ḷ} can be used immediately following a \S{k} for the
% vocalic~l, thus the root \textdeva{kḷp} can be typed “kḷp”
% rather than “kl̥p”.
%
% \item \S{ḹ} is always recognized as a long vocalic~l.
%
% \item A diaeresis can also be to break a diphthong: \N{praü} vs.\
%   \N{prau}.
%
% \item \S{n̄} is a synonym for~\S{ṅ}.
%
% \item \S{ç} is a synonym for~\S{ś}.
%
% \item \S{āi} and \S{āu} are synonyms for \S{ai} and \S{au}, respectively.
%
% \end{itemize}
%
% \begin{thebibliography}{9}
%
% \bibitem{iso15919} Anthony P. Stone.  “Transliteration of Indic
%   Scripts: How to use ISO 15919.”
%   \url{http://homepage.ntlworld.com/stone-catend/trind.htm}
%
% \bibitem{psd} Arthur A. Macdonell, \emph{A Sanskrit-English
%     Dictionary}.  Longmans, Green, and Co., London, 1893.
%
% \bibitem{sgs} Arthur A. MacDonell, \emph{A Sanskrit Grammar for
%     Students}, third edition. Oxford University Press, 1927.
%
% \bibitem{unicode7} The Unicode Consortium.  \emph{The Unicode
%     Standard: Version 7.0—Core Specification}.
%   \url{http://www.unicode.org/versions/Unicode7.0.0/UnicodeStandard-7.0.pdf}
%   (modification date 09/26/2014 03:27:41 PM; md5sum
%   aa6c89d24d7075f93259069fae273251).
%
% \bibitem{wg} William Dwight Whitney, \emph{Sanskrit Grammar:
%     Including both the Classical Language, and the older Dialects,
%     of Veda and Brahmana}.  Harvard University Press, fourteenth
%   printing, 1975.
%
% \iffalse
% \bibitem{wg} William Dwight Whitney, \emph{The Roots, Verb-Forms and
%     Primary Derivatives of the Sanskrit Language}.  Breitkopf and
%   Härtel, 1885.
% \fi
%
% \end{thebibliography}
%
% \StopEventually{}
%
% \section{TECkit header}
%
%    \begin{macrocode}
LHSName "Unicode/ISO 15919"
%<deva&!san>RHSName "Unicode/Devanagari"
%<deva&san>RHSName "Unicode/Sanskrit"
%<beng>RHSName "Unicode/Bengali"
%<gujr>RHSName "Unicode/Gujarati"
%<guru>RHSName "Unicode/Gurmukhi"
%<knda>RHSName "Unicode/Kannada"
%<mlym>RHSName "Unicode/Malayalam"
%<orya>RHSName "Unicode/Oriya"
%<sinh>RHSName "Unicode/Sinhala"
%<taml>RHSName "Unicode/Tamil"
%<telu>RHSName "Unicode/Telugu"
Version "0.01 beta"
%    \end{macrocode}
%
% \section{Overview}
%
% Conceptually, the mapping process is divided into two main phases,
% the first of which translates user input into an internal normalized
% form and the second of which translates that normalized form into
% the desired output alphabet.  So, we are actually dealing with three
% different sets of characters:
% \begin{enumerate}
%
% \item User input: The specific Latin-based Unicode characters
%   supplied by the user.  This can be a mix of upper and lower case
%   and precomposed or fully decomposed characters.
%
% \item Normalized input: During \hyperref[phase1]{Phase~1}, the user
%   input is converted into a normalized form, based on, but not
%   identical with, Unicode Normalization Form~D\@.  In broad terms,
%   all characters are downcased, and combining marks representing
%   long vowels or tone marks are separated from their base
%   characters, while combining marks designating distinct consonants
%   are combined with their base characters.  In addition, alternative
%   representations are merged and in some cases context-sensitive
%   disambiguation is performed.
%
% \item Output: During \hyperref[phase2]{Phase~2}, the normalized
%   input is converted into the desired output script.
%
% \end{enumerate}
%
% \section{Normalized input characters}
%
% \subsection{Vowels and diphthongs}
%
%    \begin{macrocode}
Define a        U+0061
Define ae       U+00E6
Define b        U+0062
Define c        U+0063
Define d        U+0064
Define e        U+0065
Define f        U+0066
Define g        U+0067
Define h        U+0068
Define i        U+0069
Define j        U+006A
Define k        U+006B
Define l        U+006C
Define m        U+006D
Define n        U+006E
Define o        U+006F
Define p        U+0070
Define q        U+0071
Define r        U+0072
Define s        U+0073
Define t        U+0074
Define u        U+0075
Define v        U+0076
Define w        U+0077
; Define x        U+0078
Define y        U+0079
Define z        U+007A
%    \end{macrocode}
%    It will be convenient to have single-character representations
%    for vocalic~r and vocalic~l, but Unicode doesn't have precomposed
%    characters for l or~r with a ring below.  So, for convenience, we
%    will map them to the PUA in our internal representation.
%    \begin{macrocode}
Define ri       U+E000 ; PUA
Define lry      U+E001 ; PUA
%    \end{macrocode}
%
%    \subsubsection{Symbols}
%
%    \begin{macrocode}
Define apos     U+0027
; Define colon    U+003A ; predefined
Define degree   U+00B0
Define bar      U+007C

Define lbrack   U+005B
Define rbrack   U+005D
%    \end{macrocode}
%
% \subsection{Combining accents}
%
% Define short names for the various combining accents we
% need.\footnote{The names tilde, macron, breve, dot\_above, and
%   diaeresis are predefined the compiler to refer to the spacing
%   variants of those accents, so I've had to choose alternate names.}
%    \begin{macrocode}
Define grave        U+0300
Define acute        U+0301
Define circumflex   U+0302
Define tildeabove   U+0303
Define macronabove  U+0304
Define breveabove   U+0306
Define dotabove     U+0307
Define dieresis     U+0308
Define vert         U+030D
Define CANDRABINDU  U+0310
Define dotbelow     U+0323
Define ringbelow    U+0325
Define ccedilla     U+0327
Define hookbelow    U+0328
Define brevebelow   U+032E
Define macronbelow  U+0331
%    \end{macrocode}
%
% \section{Output Characters}
%
%    Now define the output space.
% \subsection{Script-independent characters}
%    \begin{macrocode}
Define danda    U+0964
Define dbldanda U+0965

Define zwnj     U+200C ; ZERO WIDTH NON-JOINER
Define zwj      U+200D ; ZERO WIDTH JOINER
Define hiatus   U+0020 ; SPACE
Define nbsp     U+00A0 ; NO-BREAK SPACE

Define NOTDEF   U+FFFD
%    \end{macrocode}
%
% \subsection{Devanagari}
%
%    \begin{macrocode}
%<*deva>
; Define altcandrabindu U+0900
Define candrabindu U+0901
Define anusvara    U+0902
Define visarga     U+0903

; Define ashort    U+0904
Define asvara      U+0905
Define Asvara      U+0906

Define isvara      U+0907
Define Isvara      U+0908

Define usvara      U+0909
Define Usvara      U+090A

Define rsvara      U+090B
Define lsvara      U+090C

%<*san>
Define esvara      U+090F
Define Esvara      U+090F
Define ecandra     U+090F
%</san>
%<*!san>
Define esvara      U+090E
Define Esvara      U+090F
Define ecandra     U+090D
%</!san>

Define aisvara     U+0910

%<*san>
Define osvara      U+0913
Define Osvara      U+0913
Define ocandra     U+0913
%</san>
%<*!san>
Define osvara      U+0912
Define Osvara      U+0913
Define ocandra     U+0911
%</!san>

Define ausvara     U+0914

Define ka          U+0915
Define kha         U+0916
Define ga          U+0917
Define gha         U+0918
Define nga         U+0919

Define ca          U+091A
Define cha         U+091B
Define ja          U+091C
Define jha         U+091D
Define nya         U+091E

Define tta         U+091F
Define ttha        U+0920
Define dda         U+0921
Define ddha        U+0922
Define nna         U+0923

Define ta          U+0924
Define tha         U+0925
Define da          U+0926
Define dha         U+0927
Define na          U+0928

Define nnna        U+0929

Define pa          U+092A
Define pha         U+092B
Define ba          U+092C
Define bha         U+092D
Define ma          U+092E

Define ya          U+092F
Define ra          U+0930
Define rra         U+0931
Define la          U+0932
Define lla         U+0933
Define llla        U+0934
Define va          U+0935

Define sha         U+0936
Define ssa         U+0937
Define sa          U+0938
Define ha          U+0939

; Define oematra     U+093A
; Define OEmatra     U+093B

Define nukta       U+093C
Define avagraha    U+093D

Define Amatra      U+093E

Define imatra      U+093F
Define Imatra      U+0940

Define umatra      U+0941
Define Umatra      U+0942

Define rmatra      U+0943
Define Rmatra      U+0944

%<*san>
Define ematra       U+0947
Define Ematra       U+0947
Define ecandramatra U+0947
%</san>
%<*!san>
Define ematra       U+0946
Define Ematra       U+0947
Define ecandramatra U+0945
%</!san>

Define aimatra     U+0948

%<*san>
Define omatra       U+094B
Define Omatra       U+094B
Define ocandramatra U+094B
%</san>
%<*!san>
Define omatra       U+094A
Define Omatra       U+094B
Define ocandramatra U+0949
%</!san>

Define aumatra     U+094C

Define virama      U+094D
; Define prishthamatra U+094E
; Define awmatra   U+094F
; Define om        U+0950

Define udatta      U+0951
Define anudatta    U+0952

; Define devgrave  U+0953
; Define devacute  U+0954

; Define Ecandra   U+0955

; Define uematra   U+0956
; Define UEmatra   U+0957

%<*nukta>
Define qa          U+0958
Define khha        U+0959
Define ghha        U+095A
Define za          U+095B
Define dddha       U+095C
Define rha         U+095D
Define fa          U+095E
Define yya         U+095F
%</nukta>

%<*!nukta>
Define qa          NOTDEF
Define khha        NOTDEF
Define ghha        NOTDEF
Define za          NOTDEF
Define dddha       NOTDEF
Define rha         NOTDEF
Define fa          NOTDEF
Define yya         NOTDEF
%</!nukta>

Define Rsvara      U+0960
Define Lsvara      U+0961
Define lmatra      U+0962
Define Lmatra      U+0963

Define zero        U+0966
Define one         U+0967
Define two         U+0968
Define three       U+0969
Define four        U+096A
Define five        U+096B
Define six         U+096C
Define seven       U+096D
Define eight       U+096E
Define nine        U+096F

Define abbrev      U+0970

; Define devdot    U+0971
; Define acandra    U+0972
; Define oesvara    U+0973
; Define OEsvara    U+0974
; Define awsvara    U+0975
; Define uesvara    U+0976
; Define UEsvara    U+0977
; Define ddamarwari U+0978
; Define zha        U+0979
; Define jjya       U+097a
; Define gga        U+097b
; Define jja        U+097c
; Define devglottal U+097d
; Define ddda       U+097e
; Define bba        U+097f

%<san>Define svarita_k_i U+1CD7

Define jihvamuliya U+1CF5
Define upadhmaniya U+1CF6

Define extravyanjana dddha fa ghha khha lla llla qa rha yya za
%</deva>
%    \end{macrocode}
%
% \subsection{Bengali}
%
%    \begin{macrocode}
%<*beng>
; Define anji      U+0980
Define candrabindu U+0981
Define anusvara    U+0982
Define visarga     U+0983

Define asvara      U+0985
Define Asvara      U+0986

Define isvara      U+0987
Define Isvara      U+0988

Define usvara      U+0989
Define Usvara      U+098A

Define rsvara      U+098B
Define lsvara      U+098C

Define esvara      U+098F
Define Esvara      U+098F
Define ecandra     U+098F
Define aisvara     U+0990

Define osvara      U+0993
Define Osvara      U+0993
Define ocandra     U+0993
Define ausvara     U+0994

Define ka          U+0995
Define kha         U+0996
Define ga          U+0997
Define gha         U+0998
Define nga         U+0999

Define ca          U+099A
Define cha         U+099B
Define ja          U+099C
Define jha         U+099D
Define nya         U+099E

Define tta         U+099F
Define ttha        U+09A0
Define dda         U+09A1
Define ddha        U+09A2
Define nna         U+09A3

Define ta          U+09A4
Define tha         U+09A5
Define da          U+09A6
Define dha         U+09A7
Define na          U+09A8

Define pa          U+09AA
Define pha         U+09AB
Define ba          U+09AC
Define bha         U+09AD
Define ma          U+09AE

Define ya          U+09AF
Define ra          U+09B0
Define la          U+09B2
Define lla         NOTDEF
Define va          U+09F1

Define sha         U+09B6
Define ssa         U+09B7
Define sa          U+09B8
Define ha          U+09B9

Define nukta       U+09BC
Define avagraha    U+09BD

Define Amatra      U+09BE

Define imatra      U+09BF
Define Imatra      U+09C0

Define umatra      U+09C1
Define Umatra      U+09C2

Define rmatra      U+09C3
Define Rmatra      U+09C4

Define ematra       U+09C7
Define Ematra       U+09C7
Define ecandramatra U+09C7
Define aimatra      U+09C8

Define omatra       U+09CB
Define Omatra       U+09CB
Define ocandramatra U+09CB
Define aumatra      U+09CC

Define virama      U+09CD

Define khandata    U+09CE

Define udatta      NOTDEF
Define anudatta    NOTDEF

Define aulength    U+09D7

Define dddha       U+09DC

Define qa          U+E002 ; PUA
Define khha        U+E005 ; PUA
Define ghha        U+E007 ; PUA
Define za          U+E008 ; PUA
Define rra         NOTDEF
Define rha         U+09DD
Define fa          U+E009 ; PUA
Define yya         U+09DF

Define Rsvara      U+09E0
Define Lsvara      U+09E1
Define lmatra      U+09E2
Define Lmatra      U+09E3

Define zero        U+09E6
Define one         U+09E7
Define two         U+09E8
Define three       U+09E9
Define four        U+09EA
Define five        U+09EB
Define six         U+09EC
Define seven       U+09ED
Define eight       U+09EE
Define nine        U+09EF

Define wa          U+E010 ; PUA

; Define ramd U+09F0
; Define rald U+09F1
; Define rupeemark U+09F2
; Define rupeesign U+09F3
; Define curnumone U+09F4
; Define curnumtwo U+09F5
; Define curnumthree U+09F6
; Define curnumfour U+09F7
; Define curnumoneless U+09F8
; Define curdenosixteen U+09F9
; Define isshar U+09FA
; Define ganda mark U+09FB

Define abbrev      U+0970

Define extravyanjana dddha fa ghha khha qa rra va wa za
%</beng>
%    \end{macrocode}
%
% \subsection{Gujarati}
%
%    \begin{macrocode}
%<*gujr>
Define candrabindu U+0A81
Define anusvara    U+0A82
Define visarga     U+0A83

Define asvara      U+0A85
Define Asvara      U+0A86

Define isvara      U+0A87
Define Isvara      U+0A88

Define usvara      U+0A89
Define Usvara      U+0A8A

Define rsvara      U+0A8B
Define lsvara      U+0A8C

Define esvara      U+0A8F
Define Esvara      U+0A8F
Define ecandra     U+0A8D
Define aisvara     U+0A90

Define osvara      U+0A93
Define Osvara      U+0A93
Define ocandra     U+0A91
Define ausvara     U+0A94

Define ka          U+0A95
Define kha         U+0A96
Define ga          U+0A97
Define gha         U+0A98
Define nga         U+0A99

Define ca          U+0A9A
Define cha         U+0A9B
Define ja          U+0A9C
Define jha         U+0A9D
Define nya         U+0A9E

Define tta         U+0A9F
Define ttha        U+0AA0
Define dda         U+0AA1
Define ddha        U+0AA2
Define nna         U+0AA3

Define ta          U+0AA4
Define tha         U+0AA5
Define da          U+0AA6
Define dha         U+0AA7
Define na          U+0AA8

Define pa          U+0AAA
Define pha         U+0AAB
Define ba          U+0AAC
Define bha         U+0AAD
Define ma          U+0AAE

Define ya          U+0AAF
Define ra          U+0AB0
Define la          U+0AB2
Define lla         U+0AB3
Define va          U+0AB5

Define sha         U+0AB6
Define ssa         U+0AB7
Define sa          U+0AB8
Define ha          U+0AB9

Define nukta       U+0ABC
Define avagraha    U+0ABD

Define Amatra      U+0ABE

Define imatra      U+0ABF
Define Imatra      U+0AC0

Define umatra      U+0AC1
Define Umatra      U+0AC2

Define rmatra      U+0AC3
Define Rmatra      U+0AC4

Define ematra      U+0AC7
Define Ematra      U+0AC7
Define ecandramatra U+0AC5
Define aimatra     U+0AC8

Define omatra       U+0ACB
Define Omatra       U+0ACB
Define ocandramatra U+0AC9
Define aumatra      U+0ACC

Define virama      U+0ACD
; Define om        U+0AD0

Define udatta      NOTDEF
Define anudatta    NOTDEF

Define qa          U+E003 ; PUA
Define khha        U+E006 ; PUA
Define ghha        U+E007 ; PUA
Define za          U+E008 ; PUA
Define rra         NOTDEF
Define rha         NOTDEF
Define fa          U+E009 ; PUA
Define yya         NOTDEF

Define Rsvara      U+0AE0
Define Lsvara      U+0AE1
Define lmatra      U+0AE2
Define Lmatra      U+0AE3

Define zero        U+0AE6
Define one         U+0AE7
Define two         U+0AE8
Define three       U+0AE9
Define four        U+0AEA
Define five        U+0AEB
Define six         U+0AEC
Define seven       U+0AED
Define eight       U+0AEE
Define nine        U+0AEF

Define abbrev      U+0AF0

; Define rupeemark NOTDEF
; Define rupeesign U+0AF1

Define extravyanjana fa ghha khha qa za
%</gujr>
%    \end{macrocode}
%
% \subsection{Gurmukhi}
%
%    \begin{macrocode}
%<*guru>
Define candrabindu U+0A01
Define anusvara    U+0A02
Define visarga     U+0A03

Define asvara      U+0A05
Define Asvara      U+0A06

Define isvara      U+0A07
Define Isvara      U+0A08

Define usvara      U+0A09
Define Usvara      U+0A0A

Define rsvara      NOTDEF
Define Rsvara      NOTDEF

Define lsvara      NOTDEF
Define Lsvara      NOTDEF

Define esvara      U+0A0F
Define Esvara      U+0A0F
Define ecandra     U+0A0F
Define aisvara     U+0A10

Define osvara      U+0A13
Define Osvara      U+0A13
Define ocandra     U+0A13
Define ausvara     U+0A14

Define ka          U+0A15
Define kha         U+0A16
Define ga          U+0A17
Define gha         U+0A18
Define nga         U+0A19

Define ca          U+0A1A
Define cha         U+0A1B
Define ja          U+0A1C
Define jha         U+0A1D
Define nya         U+0A1E

Define tta         U+0A1F
Define ttha        U+0A20
Define dda         U+0A21
Define ddha        U+0A22
Define nna         U+0A23

Define ta          U+0A24
Define tha         U+0A25
Define da          U+0A26
Define dha         U+0A27
Define na          U+0A28

Define pa          U+0A2A
Define pha         U+0A2B
Define ba          U+0A2C
Define bha         U+0A2D
Define ma          U+0A2E

Define ya          U+0A2F
Define ra          U+0A30
Define la          U+0A32
Define lla         U+0A33
Define va          U+0A35

Define sha         U+0A36
Define ssa         NOTDEF
Define sa          U+0A38
Define ha          U+0A39

Define nukta       U+0A3C
Define avagraha    NOTDEF

Define Amatra      U+0A3E

Define imatra      U+0A3F
Define Imatra      U+0A40

Define umatra      U+0A41
Define Umatra      U+0A42

Define rmatra      NOTDEF
Define Rmatra      NOTDEF

Define ematra       U+0A47
Define Ematra       U+0A47
Define ecandramatra U+0A47
Define aimatra      U+0A48

Define omatra       U+0A4B
Define Omatra       U+0A4B
Define ocandramatra U+0A4B
Define aumatra      U+0A4C

Define virama      U+0A4D

Define udatta      U+0A51
Define anudatta    NOTDEF

Define qa          U+E004 ; PUA
Define khha        U+0A59
Define ghha        U+0A5A
Define za          U+0A5B
Define rra         U+0A5C
Define rha         NOTDEF
Define fa          U+0A5E
Define yya         NOTDEF

Define lmatra      NOTDEF
Define Lmatra      NOTDEF

Define zero        U+0A66
Define one         U+0A67
Define two         U+0A68
Define three       U+0A69
Define four        U+0A6A
Define five        U+0A6B
Define six         U+0A6C
Define seven       U+0A6D
Define eight       U+0A6E
Define nine        U+0A6F

Define abbrev      NOTDEF

Define tippi       U+0A70

Define addak       U+0A71
Define iri         U+0A72
Define ura         U+0A73
Define ek_onkar    U+0A74
Define yakash      U+0A75

Define extravyanjana fa ghha khha qa rra za
%</guru>
%    \end{macrocode}
%
% \subsection{Kannada}
%
%    \begin{macrocode}
%<*knda>
Define candrabindu U+0C81
Define anusvara    U+0C82
Define visarga     U+0C83

Define asvara      U+0C85
Define Asvara      U+0C86

Define isvara      U+0C87
Define Isvara      U+0C88

Define usvara      U+0C89
Define Usvara      U+0C8A

Define rsvara      U+0C8B

Define lsvara      U+0C8C

Define esvara      U+0C8E
Define Esvara      U+0C8F
Define ecandra     U+0C8F
Define aisvara     U+0C90

Define osvara      U+0C92
Define Osvara      U+0C93
Define ocandra     U+0C93
Define ausvara     U+0C94

Define ka          U+0C95
Define kha         U+0C96
Define ga          U+0C97
Define gha         U+0C98
Define nga         U+0C99

Define ca          U+0C9A
Define cha         U+0C9B
Define ja          U+0C9C
Define jha         U+0C9D
Define nya         U+0C9E

Define tta         U+0C9F
Define ttha        U+0CA0
Define dda         U+0CA1
Define ddha        U+0CA2
Define nna         U+0CA3

Define ta          U+0CA4
Define tha         U+0CA5
Define da          U+0CA6
Define dha         U+0CA7
Define na          U+0CA8

Define pa          U+0CAA
Define pha         U+0CAB
Define ba          U+0CAC
Define bha         U+0CAD
Define ma          U+0CAE

Define ya          U+0CAF
Define ra          U+0CB0
Define rra         U+0CB1
Define la          U+0CB2
Define lla         U+0CB3
Define va          U+0CB5

Define llla        U+0CDE

Define sha         U+0CB6
Define ssa         U+0CB7
Define sa          U+0CB8
Define ha          U+0CB9

Define nukta       U+0CBC
Define avagraha    U+0CBD

Define Amatra      U+0CBE

Define imatra      U+0CBF
Define Imatra      U+0CC0

Define umatra      U+0CC1
Define Umatra      U+0CC2

Define rmatra      U+0CC3
Define Rmatra      U+0CC4

Define ematra       U+0CC6
Define Ematra       U+0CC7
Define ecandramatra U+0CC7
Define aimatra      U+0CC8

Define omatra       U+0CCA
Define Omatra       U+0CCB
Define ocandramatra U+0CCB
Define aumatra      U+0CCC

Define udatta      NOTDEF
Define anudatta    NOTDEF

Define virama         U+0CCD
; Define length_mark    U+0CD5
; Define ai_length_mark U+0CD6

Define za          U+E008 ; PUA
Define fa          U+E014 ; PUA

Define Rsvara      U+0CE0
Define Lsvara      U+0CE1
Define lmatra      U+0CE2
Define Lmatra      U+0CE3

Define zero        U+0CE6
Define one         U+0CE7
Define two         U+0CE8
Define three       U+0CE9
Define four        U+0CEA
Define five        U+0CEB
Define six         U+0CEC
Define seven       U+0CED
Define eight       U+0CEE
Define nine        U+0CEF

Define abbrev      NOTDEF

Define jihvamuliya U+0CF1
Define upadhmaniya U+0CF2

Define extravyanjana fa llla za 
%</knda>
%    \end{macrocode}
%
% \subsection{Malayalam}
%
%    \begin{macrocode}
%<*mlym>
Define candrabindu U+0D01
Define anusvara    U+0D02
Define visarga     U+0D03

Define asvara      U+0D05
Define Asvara      U+0D06

Define isvara      U+0D07
Define Isvara      U+0D08

Define usvara      U+0D09
Define Usvara      U+0D0A

Define rsvara      U+0D0B
Define lsvara      U+0D0C

Define esvara      U+0D0E
Define Esvara      U+0D0F
Define ecandra     U+0D0F
Define aisvara     U+0D10

Define osvara      U+0D12
Define Osvara      U+0D13
Define ocandra     U+0D13
Define ausvara     U+0D14

Define ka          U+0D15
Define kha         U+0D16
Define ga          U+0D17
Define gha         U+0D18
Define nga         U+0D19

Define ca          U+0D1A
Define cha         U+0D1B
Define ja          U+0D1C
Define jha         U+0D1D
Define nya         U+0D1E

Define tta         U+0D1F
Define ttha        U+0D20
Define dda         U+0D21
Define ddha        U+0D22
Define nna         U+0D23

Define ta          U+0D24
Define tha         U+0D25
Define da          U+0D26
Define dha         U+0D27
Define na          U+0D28

Define nnna        U+0D29

Define pa          U+0D2A
Define pha         U+0D2B
Define ba          U+0D2C
Define bha         U+0D2D
Define ma          U+0D2E

Define ya          U+0D2F
Define ra          U+0D30
Define rra         U+0D31
Define la          U+0D32
Define lla         U+0D33
Define llla        U+0D34
Define va          U+0D35

Define sha         U+0D36
Define ssa         U+0D37
Define sa          U+0D38
Define ha          U+0D39

Define ttta        U+0D3A

Define avagraha    U+0D3D

Define Amatra      U+0D3E

Define imatra      U+0D3F
Define Imatra      U+0D40

Define umatra      U+0D41
Define Umatra      U+0D42

Define rmatra      U+0D43
Define Rmatra      U+0D44

Define ematra       U+0D46
Define Ematra       U+0D47
Define ecandramatra U+0D47
Define aimatra      U+0D48

Define omatra       U+0D4A
Define Omatra       U+0D4B
Define ocandramatra U+0D4B
Define aumatra      U+0D4C

Define virama      U+0D4D

Define dot_reph    U+0D4E

Define udatta      U+0D51
Define anudatta    U+0D52

; Define au_length_mark U+0D57

Define fa          NOTDEF

Define Rsvara      U+0D60
Define Lsvara      U+0D61
Define lmatra      U+0D62
Define Lmatra      U+0D63

Define zero        U+0D66
Define one         U+0D67
Define two         U+0D68
Define three       U+0D69
Define four        U+0D6A
Define five        U+0D6B
Define six         U+0D6C
Define seven       U+0D6D
Define eight       U+0D6E
Define nine        U+0D6F

; Define ten         U+0D70
; Define hundred     U+0D71
; Define thousand    U+0D72

; Define quarter     U+0D73
; Define half        U+0D74
; Define three_quarters U+0D75

; Define date_mark   U+0D79

Define abbrev      NOTDEF

Define chillu_nn U+0D7A
Define chillu_n  U+0D7B
Define chillu_rr U+0D7C
Define chillu_l  U+0D7D
Define chillu_ll U+0D7E
Define chillu_k  U+0D7F

Define extravyanjana llla ttta
%</mlym>
%    \end{macrocode}
%
% \subsection{Oriya}
%
%    \begin{macrocode}
%<*orya>
Define candrabindu U+0B01
Define anusvara    U+0B02
Define visarga     U+0B03

Define asvara      U+0B05
Define Asvara      U+0B06

Define isvara      U+0B07
Define Isvara      U+0B08

Define usvara      U+0B09
Define Usvara      U+0B0A

Define rsvara      U+0B0B
Define lsvara      U+0B0C

Define esvara      U+0B0F
Define Esvara      U+0B0F
Define ecandra     U+0B0F
Define aisvara     U+0B10

Define osvara      U+0B13
Define Osvara      U+0B13
Define ocandra     U+0B13
Define ausvara     U+0B14

Define ka          U+0B15
Define kha         U+0B16
Define ga          U+0B17
Define gha         U+0B18
Define nga         U+0B19

Define ca          U+0B1A
Define cha         U+0B1B
Define ja          U+0B1C
Define jha         U+0B1D
Define nya         U+0B1E

Define tta         U+0B1F
Define ttha        U+0B20
Define dda         U+0B21
Define ddha        U+0B22
Define nna         U+0B23

Define ta          U+0B24
Define tha         U+0B25
Define da          U+0B26
Define dha         U+0B27
Define na          U+0B28

Define pa          U+0B2A
Define pha         U+0B2B
Define ba          U+0B2C
Define bha         U+0B2D
Define ma          U+0B2E

Define ya          U+0B2F
Define ra          U+0B30
Define la          U+0B32
Define lla         U+0B33
Define va          U+0B35

Define sha         U+0B36
Define ssa         U+0B37
Define sa          U+0B38
Define ha          U+0B39

Define nukta       U+0B3C
Define avagraha    U+0B3D

Define Amatra      U+0B3E

Define imatra      U+0B3F
Define Imatra      U+0B40

Define umatra      U+0B41
Define Umatra      U+0B42

Define rmatra      U+0B43
Define Rmatra      U+0B44

Define ematra       U+0B47
Define Ematra       U+0B47
Define ecandramatra U+0B47
Define aimatra      U+0B48

Define omatra       U+0B4B
Define Omatra       U+0B4B
Define ocandramatra U+0B4B
Define aumatra      U+0B4C

Define virama      U+0B4D

Define udatta      NOTDEF
Define anudatta    NOTDEF

; Define ai_length_mark U+0B56
; Define au_length_mark U+0B57

Define rra         U+0B5C
Define rha         U+0B5D
Define yya         U+0B5F

Define fa          NOTDEF

Define Rsvara      U+0B60
Define Lsvara      U+0B61
Define lmatra      U+0B62
Define Lmatra      U+0B63

Define zero        U+0B66
Define one         U+0B67
Define two         U+0B68
Define three       U+0B69
Define four        U+0B6A
Define five        U+0B6B
Define six         U+0B6C
Define seven       U+0B6D
Define eight       U+0B6E
Define nine        U+0B6F

Define abbrev      NOTDEF

Define isshar           U+0B70
Define wa               U+0B71
Define quarter          U+0B72
Define half             U+0B73
Define three_quarters   U+0B74
Define sixteenth        U+0B75
Define eighth           U+0B76
Define three_sixteenths U+0B77

Define extravyanjana rha rra wa yya
%</orya>
%    \end{macrocode}
%
% \subsection{Sinhala}
%
%    \begin{macrocode}
%<*sinh>
Define candrabindu NOTDEF
Define anusvara    U+0D82
Define visarga     U+0D83

Define asvara      U+0D85
Define Asvara      U+0D86

Define aesvara     U+0D87
Define AEsvara     U+0D88

Define isvara      U+0D89
Define Isvara      U+0D8A

Define usvara      U+0D8B
Define Usvara      U+0D8C

Define rsvara      U+0D8D
Define Rsvara      U+0D8E

Define lsvara      U+0D8F
Define Lsvara      U+0D90

Define esvara      U+0D91
Define Esvara      U+0D92
Define ecandra     U+0D92
Define aisvara     U+0D93

Define osvara      U+0D94
Define Osvara      U+0D95
Define ocandra     U+0D95
Define ausvara     U+0D96

Define ka          U+0D9A
Define kha         U+0D9B
Define ga          U+0D9C
Define gha         U+0D9D
Define nga         U+0D9E
Define nnga        U+0D9F

Define nnna        U+E011 ; PUA

Define ca          U+0DA0
Define cha         U+0DA1
Define ja          U+0DA2
Define jha         U+0DA3
Define nya         U+0DA4
Define jnya        U+0DA5
Define nyja        U+0DA6

Define tta         U+0DA7
Define ttha        U+0DA8
Define dda         U+0DA9
Define ddha        U+0DAA
Define nna         U+0DAB
Define nndda       U+0DAC

Define ta          U+0DAD
Define tha         U+0DAE
Define da          U+0DAF
Define dha         U+0DB0
Define na          U+0DB1

Define nda         U+0DB3

Define pa          U+0DB4
Define pha         U+0DB5
Define ba          U+0DB6
Define bha         U+0DB7
Define ma          U+0DB8
Define mba         U+0DB9

Define ya          U+0DBA
Define ra          U+0DBB
Define rra         U+E012 ; PUA
Define la          U+0DBD
Define va          U+0DC0

Define sha         U+0DC1
Define ssa         U+0DC2
Define sa          U+0DC3
Define ha          U+0DC4
Define lla         U+0DC5
Define fa          U+0DC6

Define llla        U+E013 ; PUA

Define avagraha    NOTDEF

Define virama      U+0DCA

Define Amatra      U+0DCF

Define aematra     U+0DD0
Define AEmatra     U+0DD1

Define imatra      U+0DD2
Define Imatra      U+0DD3

Define umatra      U+0DD4
Define Umatra      U+0DD6

Define rmatra      U+0DD8

Define ematra       U+0DD9
Define Ematra       U+0DDA
Define ecandramatra U+0DDA
Define aimatra      U+0DDB

Define omatra       U+0DDC
Define Omatra       U+0DDD
Define ocandramatra U+0DDD
Define aumatra      U+0DDE

Define lmatra      U+0DDF

Define udatta      NOTDEF
Define anudatta    NOTDEF

Define zero        U+0DE6
Define one         U+0DE7
Define two         U+0DE8
Define three       U+0DE9
Define four        U+0DEA
Define five        U+0DEB
Define six         U+0DEC
Define seven       U+0DED
Define eight       U+0DEE
Define nine        U+0DEF

Define Rmatra      U+0DF2
Define Lmatra      U+0DF3

Define kunddaliya  U+0DF4

Define abbrev      NOTDEF

Define extravyanjana fa lla llla mba nda nndda nnga nnna nyja rra
%</sinh>
%    \end{macrocode}
%
% \subsection{Tamil}
%
%    \begin{macrocode}
%<*taml>
Define candrabindu NOTDEF
Define anusvara    U+0B82
Define visarga     U+0B83

Define asvara      U+0B85
Define Asvara      U+0B86

Define isvara      U+0B87
Define Isvara      U+0B88

Define usvara      U+0B89
Define Usvara      U+0B8A

Define rsvara      NOTDEF
Define lsvara      NOTDEF

Define esvara      U+0B8E
Define Esvara      U+0B8F
Define ecandra     U+0B8F
Define aisvara     U+0B90

Define osvara      U+0B92
Define Osvara      U+0B93
Define ocandra     U+0B93
Define ausvara     U+0B94

Define ka          U+0B95
Define kha         NOTDEF
Define ga          NOTDEF
Define gha         NOTDEF
Define nga         U+0B99

Define ca          U+0B9A
Define cha         NOTDEF
Define ja          U+0B9C
Define jha         NOTDEF
Define nya         U+0B9E

Define tta         U+0B9F
Define ttha        NOTDEF
Define dda         NOTDEF
Define ddha        NOTDEF
Define nna         U+0BA3

Define ta          U+0BA4
Define tha         NOTDEF
Define da          NOTDEF
Define dha         NOTDEF
Define na          U+0BA8

Define nnna        U+0BA9

Define pa          U+0BAA
Define pha         NOTDEF
Define ba          NOTDEF
Define bha         NOTDEF
Define ma          U+0BAE

Define ya          U+0BAF
Define ra          U+0BB0
Define rra         U+0BB1
Define la          U+0BB2
Define lla         U+0BB3
Define llla        U+0BB4
Define va          U+0BB5

Define sha         U+0BB6
Define ssa         U+0BB7
Define sa          U+0BB8
Define ha          U+0BB9

Define avagraha    NOTDEF

Define Amatra      U+0BBE

Define imatra      U+0BBF
Define Imatra      U+0BC0

Define umatra      U+0BC1
Define Umatra      U+0BC2

Define rmatra      NOTDEF
Define Rmatra      NOTDEF

Define ematra       U+0BC6
Define Ematra       U+0BC7
Define ecandramatra U+0BC7
Define aimatra      U+0BC8

Define omatra       U+0BCA
Define Omatra       U+0BCB
Define ocandramatra U+0BCB
Define aumatra      U+0BCC

Define virama      U+0BCD

Define om          U+0BD0

Define udatta      NOTDEF
Define anudatta    NOTDEF

; Define au_length_mark U+0BD7

Define fa          NOTDEF

Define Rsvara      NOTDEF
Define Lsvara      NOTDEF
Define lmatra      NOTDEF
Define Lmatra      NOTDEF

Define zero        U+0BE6
Define one         U+0BE7
Define two         U+0BE8
Define three       U+0BE9
Define four        U+0BEA
Define five        U+0BEB
Define six         U+0BEC
Define seven       U+0BED
Define eight       U+0BEE
Define nine        U+0BEF

Define abbrev      NOTDEF

; Define ten         U+0BF0
; Define hundred     U+0BF1
; Define thousand    U+0BF2

; Define daysign     U+0BF3
; Define monthsign   U+0BF4
; Define yearsign    U+0BF5
; Define debitsign   U+0BF6
; Define creditsign  U+0BF7
; Define asabovesign U+0BF8
; Define rupeesign   U+0BF9
; Define numbersign  U+0BFA

Define extravyanjana llla 
%</taml>
%    \end{macrocode}
%
% \subsection{Telugu}
%
%    \begin{macrocode}
%<*telu>
Define candrabindu U+0C00
Define altcandrabindu U+0C01
Define anusvara    U+0C02
Define visarga     U+0C03

Define asvara      U+0C05
Define Asvara      U+0C06

Define isvara      U+0C07
Define Isvara      U+0C08

Define usvara      U+0C09
Define Usvara      U+0C0A

Define rsvara      U+0C0B
Define lsvara      U+0C0C

Define esvara      U+0C0E
Define Esvara      U+0C0F
Define ecandra     U+0C0F
Define aisvara     U+0C10

Define osvara      U+0C12
Define Osvara      U+0C13
Define ocandra     U+0C13
Define ausvara     U+0C14

Define ka          U+0C15
Define kha         U+0C16
Define ga          U+0C17
Define gha         U+0C18
Define nga         U+0C19

Define ca          U+0C1A
Define cha         U+0C1B
Define ja          U+0C1C
Define jha         U+0C1D
Define nya         U+0C1E

Define tta         U+0C1F
Define ttha        U+0C20
Define dda         U+0C21
Define ddha        U+0C22
Define nna         U+0C23

Define ta          U+0C24
Define tha         U+0C25
Define da          U+0C26
Define dha         U+0C27
Define na          U+0C28

Define pa          U+0C2A
Define pha         U+0C2B
Define ba          U+0C2C
Define bha         U+0C2D
Define ma          U+0C2E

Define ya          U+0C2F
Define ra          U+0C30
Define rra         U+0C31
Define la          U+0C32
Define lla         U+0C33
Define llla        U+0C34
Define va          U+0C35

Define sha         U+0C36
Define ssa         U+0C37
Define sa          U+0C38
Define ha          U+0C39

Define avagraha    U+0C3D

Define Amatra      U+0C3E

Define imatra      U+0C3F
Define Imatra      U+0C40

Define umatra      U+0C41
Define Umatra      U+0C42

Define rmatra      U+0C43
Define Rmatra      U+0C44

Define ematra       U+0C46
Define Ematra       U+0C47
Define ecandramatra U+0C47
Define aimatra      U+0C48

Define omatra       U+0C4A
Define Omatra       U+0C4B
Define ocandramatra U+0C4B
Define aumatra      U+0C4C

Define virama      U+0C4D

Define udatta      NOTDEF
Define anudatta    NOTDEF

; Define length_mark    U+0C55
; Define ai_length_mark U+0C56

Define tsa         U+0C58
Define dza         U+0C59

Define fa          NOTDEF

Define Rsvara      U+0C60
Define Lsvara      U+0C61
Define lmatra      U+0C62
Define Lmatra      U+0C63

Define zero        U+0C66
Define one         U+0C67
Define two         U+0C68
Define three       U+0C69
Define four        U+0C6A
Define five        U+0C6B
Define six         U+0C6C
Define seven       U+0C6D
Define eight       U+0C6E
Define nine        U+0C6F

Define abbrev      NOTDEF

Define zero1       U+0C78
Define one1        U+0C79
Define two1        U+0C7A
Define three1      U+0C7B

Define one2        U+0C7C
Define two2        U+0C7D
Define three2      U+0C7E

Define tuumu       U+0C7F

Define extravyanjana dza llla tsa
%</telu>
%    \end{macrocode}
%
% \section{Phase 1: Normalization}
% \label{phase1}
%
% By first normalizing the input, we can simplify the transliteration
% rules without sacrificing flexibility on the input side.
%
% \subsection{Normalization Pass 1: Decompose compound characters.}
%
% Separate diacritics from their base letters and sort them into a
% standard order.  This ensures that, for example, all long vowels are
% encoded as base letter + macron and not by a precomposed character.
%
%    \begin{macrocode}
pass(NFD)
%    \end{macrocode}
%
% \subsection{Normalization Pass 2: Downcase base characters.}
%
% Convert all uppercase letters to lowercase.
%
%    \begin{macrocode}
pass(Unicode)

UniClass[upper] = (U+0041 .. U+005A) ; A..Z
UniClass[lower] = (U+0061 .. U+007A) ; a..z

[upper] > [lower]
%    \end{macrocode}
%
% This is also a good time to delete optional characters.
%    \begin{macrocode}
lbrack .* rbrack >
%    \end{macrocode}
%
%    \subsection{Normalization Pass 3: Special cases and alternate
%    transcriptions.}
%
% \S{l̥} and \S{r̥} always represent vocalic~l and vocalic~r.
%    \begin{macrocode}
pass(Unicode)

l ringbelow > lry
r ringbelow > ri
%    \end{macrocode}
%    For compatibility with IAST and common practice, allow \S{ṛ} as a
%    synonym for vocalic~r in Sanskrit.
%    \begin{macrocode}
%<san>r dotbelow  > ri
%    \end{macrocode}
%    Similarly, accept \S{ḷ} as a synonym for vocalic~l in an
%    appropriate context.  Specifically, as per section~26 of
%    Whitney's grammar, vocalic~l is only found in derivatives of the
%    rare root \S{kḷp} and (as far as I can tell) it will always
%    immediately following a~\S{k}.  The following rule allows \S{ḷ}
%    to be used in those contexts.  (See “Transliteration pass 1.5”
%    below for another way to disambiguate dual uses of~\S{ḷ}.)
%    \begin{macrocode}
%<san>l dotbelow / k _ > lry
%    \end{macrocode}
%    \S{ḹ} is always a vowel.
%    \begin{macrocode}
%<san>l dotbelow / _  macronabove > lry
%    \end{macrocode}
%    Allow \S{āi} and \S{āu} as synonyms for \S{ai} and \S{au}.
%    \begin{macrocode}
%<san>a macronabove / _ (i | u) > a
%    \end{macrocode}
%    Recognize some older transcription conventions used in, for
%    example, works by Macdonell or Whitney.
%    \begin{macrocode}
%<*san>
n macronabove <> n dotabove
c ccedilla <> s acute
circumflex <> macronabove
%</san>
%    \end{macrocode}
%
%    \begin{macrocode}
%<*san>
a macronbelow u macronbelow > a u macronbelow
a macronbelow i macronbelow > a i macronbelow
%</san>
%    \end{macrocode}
%
%    Diereses are used to keep “i” or “u” from forming diphthongs with
%    a preceding “a”.
%    \begin{macrocode}
i dieresis > zwnj i
u dieresis > zwnj u
%    \end{macrocode}
%
%    In NFD, marks under the base character are encoded before marks
%    above the character, but we need the long vowel mark and the
%    candribindu to come first.
%    \begin{macrocode}
macronbelow macronabove > macronabove macronbelow
macronbelow CANDRABINDU > CANDRABINDU macronbelow
%    \end{macrocode}
%
% \section{Sanskrit External Vowel Sandhi}
%
% This section was inspired by Macdonell's practice of writing, for
% example,
% \begin{quote}
% \N{adhi‿ūḍha}, \emph{pp.} (√ūh) placed upon (\emph{lc.}).
% \end{quote}
% in his Sanskrit-English Dictionary to illustrate the action of vowel
% sandhi in compound words.  With these rules, you can use U+203F
% UNDERTIE to request vowel coalescence in the Devanagari text:
% \begin{verbatim}
%     \N{adhi‿ūḍha}, ...\end{verbatim}

% I've attempted to implement all of sections 19–22 of \emph{A
% Sanskrit Grammar for Students}~\cite{sgs}.  The rules below are
% quoted from there, with minor adaptations and elisions.  The order
% and division into passes is chosen to ensure that, for example, the
% digraphs \S{ai} and \S{au} are processed before any of the patterns
% beginning with \S{i} or~\S{u}.

%    \begin{macrocode}
%<*san>
pass(Unicode)

Define coalesce U+203F ; UNDERTIE
%    \end{macrocode}
%
%    For simplicity (and because Macdonell doesn't display them in the
%    Devanagari in his dictionary), drop all accents.
%    \begin{macrocode}
UniClass[vaccent] = (acute grave macronbelow vert)

[vaccent] / _ coalesce                  >
[vaccent] /   coalesce _                >
[vaccent] /   coalesce . macronabove? _ >
%    \end{macrocode}
%
%    SGS §21a: The Guṇa vowels \S{e} and~\S{o} remain unchanged
%    before~\S{a}, which is elided:
%    \begin{macrocode}
coalesce a / (e | o) _ > apos
%    \end{macrocode}
%    NB: \S{e} or \S{o} followed \S{ā} is handled by the next rule
%    because more specific (i.e., longer) rules take precedence.
%
%    SGS §21b: The Guṇa vowels \S{e} and~\S{o} become \S{a} (through
%    \S{ay} and~\S{av}, which drop the semivowel) before every other
%    vowel or diphthong:
%    \begin{macrocode}
UniClass[othervowel] = (i u ri lry e o)

( e | o ) coalesce / _ a macronabove > a hiatus
( e | o ) coalesce / _ [othervowel]  > a hiatus
%    \end{macrocode}
%
% SGS §22: The Vṛddhi vowels ai and au respectively become ā and āv
% before every vowel (or diphthong):
%    \begin{macrocode}
a i coalesce / _ (a | [othervowel]) > a macronabove hiatus
a u coalesce / _ (a | [othervowel]) > a macronabove v
%    \end{macrocode}
% SGS §22a: The (secondary) hiatus occasioned by the dropping of y and
% v in the above three cases (21b and 22) remains.
%
%    \begin{macrocode}
pass(Unicode)
%    \end{macrocode}
%
% SGS §19a: a and ā coalesce with a following simple liquid vowel to Guṇa:
%    \begin{macrocode}
a macronabove? coalesce i   macronabove? > e
a macronabove? coalesce u   macronabove? > o
a macronabove? coalesce ri  macronabove? > a r
a macronabove? coalesce lry macronabove? > a l
%    \end{macrocode}
%
% SGS §19b: a and ā coalesce with Guṇa vowels to Vṛddhi
%
% SGS §19c: a and ā are absorbed by Vṛddhi vowels
%    \begin{macrocode}
a macronabove? coalesce (e | a i) > a i
a macronabove? coalesce (o | a u) > a u
%    \end{macrocode}
%
% Other cases of §19b,c are digraphs handled by §18 below.
%
% SGS §20: A simple liquid vowel followed by any other vowel or by a
% diphthong is changed to its semivowel:
%    \begin{macrocode}
UniClass[anyvowel] = (a i u ri lry e o)
UniClass[liquid]   = (  i u ri lry)
UniClass[sv]       = (  y v r  l)

[liquid] macronabove? coalesce / _ [anyvowel] > [sv]
%    \end{macrocode}
%
% SGS §18: If the same simple vowel (short or long) occurs at the end
% and beginning of words, the result is a long vowel:
%    \begin{macrocode}
a   macronabove? coalesce a   macronabove? > a   macronabove
i   macronabove? coalesce i   macronabove? > i   macronabove
u   macronabove? coalesce u   macronabove? > u   macronabove
ri  macronabove? coalesce ri  macronabove? > ri  macronabove
lry macronabove? coalesce lry macronabove? > lry macronabove
%    \end{macrocode}
%
% \subsection{Irregular sandhi}
%
% U+2054 INVERTED UNDERTIE can be used to invoke an alternate set of
% rules, for example,
% \begin{quote}
% \N{upa⁔oṣaṇa}, \emph{n.} fasting.
% \end{quote}
% where \N{a} and \N{o} have coalesced to form \N{o} rather than the
% regular \N{au}.
%    \begin{macrocode}
Define altcoalesce U+2054 ; INVERTED UNDERTIE

a altcoalesce o > o
%</san>
%    \end{macrocode}
% (This is the only exception I've come across so far; if there are
% others, I'll add them.)
%
% \section{Phase 2: Transliteration}
% \label{phase2}
%
% \subsection{Transliteration Pass 1: Transliterate blindly.}
%
% After this step, all characters will be replaced by their
% approximate Devanagari equivalents, but the result will not be
% well-formed.
%    \begin{macrocode}
pass(Unicode)
%    \end{macrocode}
% \subsubsection{Numerals}
%    \begin{macrocode}
U+0030 > zero
U+0031 > one
U+0032 > two
U+0033 > three
U+0034 > four
U+0035 > five
U+0036 > six
U+0037 > seven
U+0038 > eight
U+0039 > nine
%    \end{macrocode}
%
% \subsubsection{Vowels}
%    \begin{macrocode}
UniClass[short_in]  = (a      i      u      ri     lry)
UniClass[short_out] = (asvara isvara usvara rsvara lsvara)
UniClass[long_out]  = (Asvara Isvara Usvara Rsvara Lsvara)

[short_in] macronabove > [long_out]

[short_in] > [short_out]

%<mlym>u breveabove > virama ; ???
%<!mlym>u breveabove > usvara

e             > esvara
e macronabove > Esvara
a i           > aisvara

e circumflex > ecandra
o circumflex > ocandra

o             > osvara
o macronabove > Osvara
a u           > ausvara

%<sinh>ae > aesvara
%<sinh>ae macronabove > AEsvara

%<telu>n breveabove > altcandrabindu ; ???

%<deva|knda>h macronbelow > jihvamuliya
%<deva|knda>h brevebelow  > upadhmaniya

%<taml>k macronbelow > visarga

apos > avagraha
%    \end{macrocode}
%
% \subsubsection{Tone marks}
%
% \iffalse
% RV = Ṛg-Veda
% MS = Maitrāyaṇī Saṃhitā
% KS = Kāṭhaka Saṃhitā
%
% Name     Meaning        Pitch   RV           MS      KS
% udātta   “raised”       high    <unmarked>   VERT    VERT
% svarita  “sounded”      middle  VERT
% anudātta “not raised”   low     MACRONBELOW
% \fi
%    \begin{macrocode}
vert        > udatta
macronbelow > anudatta
%<san>hookbelow   > svarita_k_i
%    \end{macrocode}
%
% Delete non-Vedic accents.
%    \begin{macrocode}
grave >
acute >
tildeabove >
%    \end{macrocode}
%
%    \subsubsection{Bindus and visargas}
%    \begin{macrocode}
m dotabove > anusvara
h dotbelow > visarga
%    \end{macrocode}
%    The International Alphabet of Sanskrit Transliteration uses \S{ṃ}
%    instead of \S{ṁ} for the anusvara.  Since \S{ṃ} is only otherwise
%    used for the tippi in Gurmukhi, allow it as a synonym for the
%    anusvara everywhere else.
%    \begin{macrocode}
%<guru>m dotbelow > tippi
%<!guru>m dotbelow > anusvara
%    \end{macrocode}
%    \begin{macrocode}
m CANDRABINDU > candrabindu
n CANDRABINDU > candrabindu
%    \end{macrocode}
%
% \subsubsection{Consonants (no conjuncts)}
%
%    \subsubsection{Nuktas}
%
%    \begin{macrocode}
%<*nukta>
q          <> qa
k macronbelow h macronbelow > khha
g dotabove > ghha
%<!telu>z > za
f > fa
%</nukta>
%<knda>z > za
%<knda|sinh>f > fa
%<telu>z > dza
%<beng|orya>w > wa
%    \end{macrocode}
%
%    \subsubsection{Velar}
%
%    \begin{macrocode}
k          <> ka
k h        <> kha
g          <> ga
g h        <> gha
n dotabove <> nga
%<sinh>n breveabove g > nnga
%    \end{macrocode}
%
%    \subsubsection{Palatal}
%
%    \begin{macrocode}
c            <> ca
%<telu>c circumflex <> tsa
c h          <> cha
j            <> ja
j h          <> jha
n tildeabove <> nya
%<sinh>n breveabove j <> nyja
%    \end{macrocode}
%
%    \subsubsection{Retroflex}
%
%    \begin{macrocode}
t dotbelow   <> tta
t dotbelow h <> ttha
d dotbelow   <> dda
d dotbelow h <> ddha
n dotbelow   <> nna

%<sinh>n breveabove d <> nda
%<sinh>n breveabove d dotbelow <> nndda

%<deva&!san>r dotbelow > dddha
%<beng>r dotbelow > dddha
%<guru|orya>r dotbelow > rra

%<deva|beng|orya>r dotbelow h > rha
%    \end{macrocode}
%
%    \subsubsection{Dental}
%
%    \begin{macrocode}
t       <> ta
t h     <> tha
d       <> da
d h     <> dha
n       <> na
%    \end{macrocode}
%
%    \subsubsection{Labial}
%
%    \begin{macrocode}
p       <> pa
p h     <> pha
b       <> ba
b h     <> bha
m       <> ma
%<sinh>m breveabove b <> mba
%    \end{macrocode}
%
%    \subsubsection{Semi-vowels}
%
%    \subsubsection{???}
%
%    \begin{macrocode}
r macronbelow <> rra
%<mlym>t macronbelow <> ttta
%<deva|taml|mlym>n macronbelow <> nnna
%<sinh>n macronbelow <> nnna
%<deva|taml|mlym|knda|telu|sinh>l macronbelow <> llla
%    \end{macrocode}
%
%    \subsubsection{Semi-vowels}
%
%    \begin{macrocode}
y            <> ya
%<deva|beng|orya>y dotabove > yya
r breveabove > rra zwj
r            <> ra
l            <> la
v            <> va
%    \end{macrocode}
%
%    \subsubsection{Sibilants}
%
%    \begin{macrocode}
s acute    <> sha
s dotbelow <> ssa
s          <> sa
h          <> ha
%    \end{macrocode}
%
%    \subsubsection{Vedic lingual~l}
%
%    \begin{macrocode}
l dotbelow   <> lla
%<*san>
l dotbelow h <> lla ha
%</san>
%    \end{macrocode}
%
% \subsubsection{Punctuation and symbols}
%
%    \begin{macrocode}
degree  <> abbrev
bar     <> danda
bar bar <> dbldanda
%    \end{macrocode}
%
% \subsubsection{Ligature breaking}
%
% Per note~12 of Stone's \emph{vade mecum}, allow a colon to break
% ligatures.
%    \begin{macrocode}
UniClass[vyanjana] = (ka .. ha extravyanjana)

; colon / [vyanjana] _ [vyanjana]  <> virama zwj

colon <> zwj
%    \end{macrocode}
%
% \subsection{Transliteration Pass 1.5}
%
% Adding an additional pass here would enable automatic disambiguation
% between \S{ḷ} representing lingual~l and \S{ḷ} representing
% vocalic~l to be made based on the phonetic environment (see section
% 54 of Whitney's grammar), but only at the cost of making it more
% complicated to type an isolated lingual~l.  It hardly seems worth
% it.
%    \begin{macrocode}
; pass(Unicode)
;
; UniClass[vyanjana] = (ka .. ha)
;
; lla / [vyanjana] _ > lsvara
; lla / # _ > lsvara
; lla / _ # > lsvara
%    \end{macrocode}
%
%    \begin{macrocode}
pass(unicode)
%    \end{macrocode}
%
%    Per rule R10 in section~12.1) of Unicode~7.0~\cite{unicode7},
%    vedic accents (“svaras”) should be encoded following bindus.
%    This is necessary to get correct rendering in, e.g., XeTeX
%    (luaTeX is more forgiving).
%    \begin{macrocode}
UniClass[svaras] = (udatta anudatta)
UniClass[bindus] = (anusvara candrabindu)

[svaras]=v1 [bindus]=b1 > @b1 @v1
%    \end{macrocode}
%
% \subsection{Transliteration Pass 2: Attach vowels to consonants.}
%
% At this stage all vowels are still represented as independent
% vowels.  This step will convert vowels following consonants into
% dependent vowel signs.
%
%    \begin{macrocode}
pass(Unicode)

UniClass[vyanjana] = (ka .. ha extravyanjana)

UniClass[svara] = (         Asvara \
                   isvara   Isvara \
                   usvara   Usvara \
                   rsvara   Rsvara \
                   lsvara   Lsvara \
                   ecandra esvara Esvara \
                   aisvara \
                   ocandra osvara Osvara \
                   ausvara)

UniClass[matra] = (       Amatra \
                   imatra Imatra \
                   umatra Umatra \
                   rmatra Rmatra \
                   lmatra Lmatra \
                   ecandramatra ematra Ematra \
                   aimatra \
                   ocandramatra omatra Omatra \
                   aumatra)

[svara] / [vyanjana] _ > [matra]
%    \end{macrocode}
%
%    Attach isolated vowel signs and other combining marks to a
%    non-breaking space (cf.\ The Unicode Standard 7.0, page 60,
%    “Exhibiting Nonspacing Marks in Isolation”.  HarfBuzz currently
%    requires this for proper rendering.
%
%    NB: visarga is a \emph{spacing} combiner.
%    \begin{macrocode}
UniClass[combiners] = ([matra] \
                       candrabindu anusvara visarga \
                       udatta anudatta \
%<san>                 svarita_k_i \
)

[combiners]=c1 / # _ > nbsp @c1
%    \end{macrocode}
%
% \subsection{Transliteration Pass 3: Form consonant conjuncts.}
%
%    \begin{macrocode}
pass(Unicode)

UniClass[vyanjana] = (ka .. ha extravyanjana)

[vyanjana] / _ [vyanjana] > [vyanjana] virama
%    \end{macrocode}
%
% \subsection{Transliteration Pass 4: Add viramas.}
%
%    \begin{macrocode}
pass(Unicode)

UniClass[vyanjana] = (ka .. ha extravyanjana)

UniClass[novirama] = (asvara Amatra \
                      imatra Imatra \
                      umatra Umatra \
                      rmatra Rmatra \
                      lmatra Lmatra \
                      ematra Ematra ecandramatra aimatra \
                      omatra Omatra ocandramatra aumatra \
                      virama)

[vyanjana] / _ ^[novirama] <> [vyanjana] virama
%    \end{macrocode}
%
% \subsection{Transliteration Pass 5: Remove inherent \S{a}.}
%
%    Finally, remove the inherent short~\S{a}.
%
%    \begin{macrocode}
pass(Unicode)

UniClass[vyanjana] = (ka .. ha extravyanjana)

[vyanjana] asvara <> [vyanjana]
%    \end{macrocode}
%
% \subsection{Transliteration Pass 6: Resolve PUA characters.}
%
%    \begin{macrocode}
%<beng|gujr|guru|knda|sinh>pass(Unicode)

%<guru|gujr|beng>qa > ka nukta
%<beng|gujr>khha > kha nukta
%<beng|gujr>ghha > ga nukta
%<beng|gujr>fa > pha nukta
%<beng|gujr|knda>za > ja nukta
%<beng>wa > ba nukta
%<knda>fa > pha nukta
%<sinh>nnna > na U+002E
%<sinh>rra > ra U+002E
%<sinh>llla > lla U+002E
%    \end{macrocode}

% \endinput

% In order to remove any possible ambiguity between vocalic~l and
% Vedic lingual consonant l, the vocalic~l will be represented
% internally by a combining ring below (cf.\ the IPA).  There's no
% single Unicode character for that, however, so for convenience we
% stash it temporarily in the PUA (see normalization pass~3 for
% more on this topic).
